%% Based on a TeXnicCenter-Template by Gyorgy SZEIDL.
%%%%%%%%%%%%%%%%%%%%%%%%%%%%%%%%%%%%%%%%%%%%%%%%%%%%%%%%%%%%%

%------------------------------------------------------------
%
\documentclass{amsart}
%
%----------------------------------------------------------
% This is a sample document for the AMS LaTeX Article Class
% Class options
%        -- Point size:  8pt, 9pt, 10pt (default), 11pt, 12pt
%        -- Paper size:  letterpaper(default), a4paper
%        -- Orientation: portrait(default), landscape
%        -- Print size:  oneside, twoside(default)
%        -- Quality:     final(default), draft
%        -- Title page:  notitlepage, titlepage(default)
%        -- Start chapter on left:
%                        openright(default), openany
%        -- Columns:     onecolumn(default), twocolumn
%        -- Omit extra math features:
%                        nomath
%        -- AMSfonts:    noamsfonts
%        -- PSAMSFonts  (fewer AMSfonts sizes):
%                        psamsfonts
%        -- Equation numbering:
%                        leqno(default), reqno (equation numbers are on the right side)
%        -- Equation centering:
%                        centertags(default), tbtags
%        -- Displayed equations (centered is the default):
%                        fleqn (equations start at the same distance from the right side)
%        -- Electronic journal:
%                        e-only
%------------------------------------------------------------
% For instance the command
%          \documentclass[a4paper,12pt,reqno]{amsart}
% ensures that the paper size is a4, fonts are typeset at the size 12p
% and the equation numbers are on the right side
%
\usepackage{amsmath}%
\usepackage{amsfonts}%
\usepackage{amssymb}%
\usepackage{graphicx}
%------------------------------------------------------------
% Theorem like environments
%
\newtheorem{theorem}{Theorem}
\theoremstyle{plain}
\newtheorem{acknowledgement}{Acknowledgement}
\newtheorem{algorithm}{Algorithm}
\newtheorem{axiom}{Axiom}
\newtheorem{case}{Case}
\newtheorem{claim}{Claim}
\newtheorem{conclusion}{Conclusion}
\newtheorem{condition}{Condition}
\newtheorem{conjecture}{Conjecture}
\newtheorem{corollary}{Corollary}
\newtheorem{criterion}{Criterion}
\newtheorem{definition}{Definition}
\newtheorem{example}{Example}
\newtheorem{exercise}{Exercise}
\newtheorem{lemma}{Lemma}
\newtheorem{notation}{Notation}
\newtheorem{problem}{Problem}
\newtheorem{proposition}{Proposition}
\newtheorem{remark}{Remark}
\newtheorem{solution}{Solution}
\newtheorem{summary}{Summary}
\numberwithin{equation}{section}
%--------------------------------------------------------
\begin{document}
Abstract\\
Every life form from bacteria to plants and animals have a certain characteristic size. However, there is evidence that evolution has selected for different sizes in different species and can continue this selection. We know that after mass extinctions typically the average size of animals is significantly reduced (pleistocene mega-fauna extinction, dinosaur extinction, etc...) and then slowly start to increase again. We also know that for many organism is possible to slowly adapt changing their size (like some families of dinosaurs evolving into modern birds, or isolated elephant populations evolving into dwarf species).
\\
The main objective of this study is to provide a simple model that could explain the relation between size of an organism and its environment.
\\
All living organisms in order to stay alive consumes energy that they procure from the environment. Some organisms need more energy than other in order to survive. This may depend on the fact that they are more complex or have larger cells (procariotic vs. eucariotic cells), or the fact that they are composed of many cells (single-cell vs. multi-cellular organisms). However, it is also reasonable to assume that bigger organisms, with more cell, are able to procure for themselves a quantity of energy proportional to their size. From a first look it will then seem that the only limit to an animal size should be the availability of resources.\\
 In this paper we are going to propose a simple network model to depict how multicellular organism distribute energy through their "body". The analysis of this model and the result from our simulations will show how different size and "body"-type strategies can coexist in the same environment and also explain how a maximum size naturally emerges not only depending on the energy available, but also because increasing size maintaining efficiency is "hard" to accomplish (achieving perfect topology becomes harder) .\\
Furthermore, in the final section we will also explain how this model could be used to investigate phenomena like electrical energy distribution networks and more generally signal exchange in networks.   
\\
\end{document}